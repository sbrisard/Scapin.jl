\documentclass[oneside]{memoir}

\usepackage{./documenter}
\usepackage{./custom}

\settocdepth{section}

\title{
    {\HUGE Scapin.jl}\\
    {\Large }
}
\author{Sébastien Brisard <sbrisard@users.noreply.github.com> and contributors}

\begin{document}

\frontmatter
\maketitle
\clearpage
\tableofcontents

\mainmatter



\part{Home}




\hypertarget{4654847035117066990}{}


\part{Scapin}



Documentation for \href{https://github.com/sbrisard/Scapin.jl}{Scapin}.



\hypertarget{14784546651712552159}{}


\chapter{Theory}


\begin{itemize}
\item \hyperlink{12321510312768487985}{Nomenclature}
\item \hyperlink{13734652986780424843}{Continuous Green operators}
\begin{itemize}
\item \hyperlink{11901743907716800672}{On Fourier series}
\item \hyperlink{6549129120501784746}{Conductivity}
\item \hyperlink{10340496927750756388}{Elasticity}
\item \hyperlink{16861761838597320957}{Hyperelasticity}
\end{itemize}
\item \hyperlink{1050018100428169985}{Discrete Green operators}
\begin{itemize}
\item \hyperlink{5250349235821641230}{On the discrete Fourier transform}
\item \hyperlink{3470771881996884583}{The approximation space}
\item \hyperlink{11094457950407373392}{Discretizations of the Green operator}
\end{itemize}
\item \hyperlink{17831501573587717000}{On the d-dimensional brick element}
\begin{itemize}
\item \hyperlink{10162273478024495150}{Geometry of the reference brick element}
\item \hyperlink{17125853350239781481}{Shape functions}
\item \hyperlink{2492262002713312068}{Gradient and strain-displacement operators}
\item \hyperlink{15811403169534872275}{Element stiffness operator}
\end{itemize}
\item \hyperlink{2489772129592300935}{Appendix}
\begin{itemize}
\item \hyperlink{18289269374885937903}{Properties of the \(\fftfreq\) function}
\end{itemize}
\item \hyperlink{7261367053393854382}{Bibliography}
\begin{itemize}
\item \hyperlink{16002782079292950992}{Brisard, S., \& Dormieux, L. (2010)}
\item \hyperlink{17627112247925688930}{Brisard, S., \& Dormieux, L. (2012)}
\item \hyperlink{8653883748112457965}{Moulinec, H., \& Suquet, P. (1994)}
\item \hyperlink{9534833447558490031}{Moulinec, H., \& Suquet, P. (1998)}
\item \hyperlink{15745057744408808743}{Suquet, P. (1990)}
\item \hyperlink{10620780704054356859}{Wicht, D., Schneider, M., \& Böhlke, T. (2020)}
\end{itemize}
\end{itemize}


\hypertarget{15255107809304980587}{}


\chapter{API Docs}




\part{Theory}


\hypertarget{12321510312768487985}{}


\chapter{Nomenclature}



\begin{itemize}
\item \(d\): dimension of the physical space (typically \(d=2, 3\))


\item \(\Omega\): \(d\)-dimensional unit-cell


\item \(L_1,\ldots, L_d\): dimensions of the unit-cell: \(\Omega=(0, L_1)\times(0, L_2)\times\cdots\times(0, L_d)\)


\item \(\lvert\Omega\rvert=L_1L_2\cdots L_d\): volume of the unit-cell


\item \(\tuple{n}\): \(d\)-dimensional tuple of integers \(\tuple{n}=(n_1, n_2, \ldots, n_d)\)


\item \(\tuple{N}=(N_1, N_2, \ldots, N_d)\): size of the simulation grid


\item \(\lvert N\rvert=N_1N_2\cdots N_d\): total number of cells


\item \(h_i=L_i/N_i\): size of the cells (\(i=1, \ldots, d\))


\item \(\cellindices=\{0, \ldots, N_1-1\}\times\{0, \ldots, N_2-1\}\times\{0, \ldots, N_3-1\}\): set of cell indices


\item \(\Omega_{\tuple{p}}\): cells of the simulation grid (\(\tuple{p}\in\cellindices\))

\end{itemize}


\hypertarget{13734652986780424843}{}


\chapter{Continuous Green operators}



In this chapter, we discuss various boundary-value problems in a periodic setting. For each of these problems, we introduce the associated \emph{Green operator}.



\hypertarget{11901743907716800672}{}


\section{On Fourier series}



Owing to the periodic setting, the fields that are involved in the various BVPs to be discussed in this chapter are expanded in Fourier series. \(\tens T\) being a \(\Omega\)-periodic tensor field



\begin{equation*}
\begin{split}\tens T(\vec x)
=\sum_{\tuple{n}∈\integers^d}\mathcal F(\tens T)(\vec k_{\tuple{n}})
\exp(\I\vec k_{\tuple{n}}\cdot\vec x),\end{split}\end{equation*}


where \(\tuple{n}\) denotes a \(d\)-dimensional tuple of integers (see \hyperlink{11521099821733334473}{Nomenclature}). The wave vectors \(\vec k_{\tuple{n}}\) are given by



\begin{equation*}
\begin{split}\vec k_{\tuple{n}}=\frac{2\PI n_1}{L_1}\vec e_1+\frac{2\PI n_2}{L_2}\vec e_2
+\cdots+\frac{2\PI n_d}{L_d}\vec e_d,\end{split}\end{equation*}


and the Fourier coefficients of \(\tens T\) are defined as follows



\begin{equation*}
\begin{split}\mathcal F(\tens T)(\vec k)=\frac1V\int_{\vec x∈\Omega}
\tens T(\vec x)\exp(-\I\vec k\cdot\vec x)\,\D x_1\cdots\D x_d.\end{split}\end{equation*}


It is recalled that the Fourier coefficients of the gradient and divergence of \(\tens T\) can readily be computed from the Fourier coefficients of \(\tens T\)



\begin{equation*}
\begin{split}\mathcal F(\tens T\otimes\nabla)(\vec k)=\mathcal F(\tens T)(\vec k)\otimes\I\vec k
\quad\text{and}\quad
\mathcal F(\tens T\cdot\nabla)(\vec k)=\mathcal F(\tens T)(\vec k)\cdot\I\vec k.\end{split}\end{equation*}


When no confusion is possible, we will use the tilde to denote the Fourier coefficients: \(\tilde{\tens T}_\tuple{n}=\mathcal F(\tens T)(\vec k_\tuple{n})\).



\hypertarget{6549129120501784746}{}


\section{Conductivity}



\hypertarget{10340496927750756388}{}


\section{Elasticity}



We first define a few functional spaces; \(\tensors_2(\Omega)\) denotes the space of second-order, symmetric, tensor fields, with square-integrable components. Then, the space \(\tens\stresses(\Omega)\) of periodic, self-equilibrated stresses is defined as follows



\begin{equation*}
\begin{split}\tens\sigma\in\stresses(\Omega)\iff\left\{
\begin{gathered}
\tens\sigma\in\tensors_2(\Omega)\\
\tens\sigma\cdot\nabla=\vec 0\text{ a.e in }\Omega\\
\tens\sigma\cdot\vec e_i\text{ is }L_i\vec e_i
\text{-periodic for all }i=1, 2, \ldots, d\text{ (no summation),}
\end{gathered}
\right.\end{split}\end{equation*}


where the last condition expresses the periodicity of tractions in all directions parallel to the sides of the unit-cell. The space \(\tens\strains(\Omega)\) of periodic, geometrically compatible strains is defined as follows



\begin{equation*}
\begin{split}\tens\varepsilon\in\strains(\Omega)\iff\left\{
\begin{gathered}
\tens\varepsilon\in\tensors_2(\Omega)\\
\tens\varepsilon=\vec u\symotimes\nabla
\text{ a.e. in }\Omega\text{ for some vector field }\vec u\\
\vec u\text{ has square-integrable components}\\
\vec u\text{ is }\Omega\text{-periodic.}
\end{gathered}
\right.\end{split}\end{equation*}


Finally, we define the spaces of stresses and strains with zero average



\begin{equation*}
\begin{split}\stresses_0(\Omega)=\bigl\{\tens\sigma\in\stresses(\Omega),
\langle\tens\sigma\rangle=\tens0\bigr\}
\quad\text{and}\quad
\strains_0(\Omega)=\{\tens\varepsilon\in\strains(\Omega),
\langle\tens\varepsilon\rangle=\tens0\bigr\}.\end{split}\end{equation*}


We are now ready to define the periodic, fourth-order Green operator for strains \(\tens\Gamma\). Let \(\tens C\) be the homogeneous elastic stiffness of the body \(\Omega\)\footnotemark[1]. Let \(\tens\tau\in\tensors_2(\Omega)\) be a prescribed tensor field (\emph{stress-polarization}). We want to find the equilibrium state of the body \(\Omega\), subjected to the eigenstress \(\tens\tau\) and periodic boundary conditions. In other words, we want to find the solution to the following problem



\begin{equation*}
\begin{split}\text{Find }\tens\sigma\in\stresses_0(\Omega)
\text{ and }\tens\varepsilon\in\strains_0(\Omega)
\text{ such that }\tens\sigma=\tens C\dbldot\tens\varepsilon+\tens\tau
\text{ a.e. in }\Omega.\end{split}\end{equation*}


\footnotetext[1]{In other words, \(\tens C\) is a constant, fourth-order tensor with major   and minor symmetries; furthermore, \(\tens C\) is positive definite.

}


Owing to the periodic boundary conditions, we use Fourier series expansions of \(\tens\tau\), \(\tens\sigma\), \(\tens\varepsilon\) and \(\vec u\)



\begin{equation*}
\begin{split}\begin{Bmatrix}
\tens\tau(\vec x)\\
\tens\sigma(\vec x)\\
\tens\varepsilon(\vec x)\\
\vec u(\vec x)
\end{Bmatrix}
=\sum_{\tuple{n}\in\integers^d}
\begin{Bmatrix}
\tilde{\tens\tau}_\tuple{n}\\
\tilde{\tens\sigma}_\tuple{n}\\
\tilde{\tens\varepsilon}_\tuple{n}\\
\tilde{\vec u}_\tuple{n}
\end{Bmatrix}
\exp(\I \vec k_\tuple{n}\cdot\vec x).\end{split}\end{equation*}


The Fourier modes \(\tilde{\tens\sigma}_\tuple{n}\), \(\tilde{\tens\varepsilon}_\tuple{n}\) and \(\tilde{\vec u}_\tuple{n}\) solve the following equations (respectively: equilibrium, geometric compatibility, constitutive relation)



\begin{equation*}
\begin{split}\begin{gathered}
\tilde{\tens\sigma}_\tuple{n}\cdot\vec k_\tuple{n}=\vec 0\\
\tilde{\tens\varepsilon}_\tuple{n}
=\frac{\I}{2}\bigl(\tilde{\vec u}_\tuple{n}\otimes\vec k_\tuple{n}
+\vec k_\tuple{n}\otimes\tilde{\vec u}_\tuple{n}\bigr)\\
\tilde{\tens\sigma}_\tuple{n}=\tens C\dbldot\tilde{\tens\varepsilon}_\tuple{n}
+\tilde{\tens\tau}_\tuple{n}.
\end{gathered}\end{split}\end{equation*}


Plugging the third equation into the second equation, and recalling that \(\tens C\) has the minor symmetries, we find the following expression of \(\tilde{\tens\sigma}\)



\begin{equation*}
\begin{split}\tilde{\tens\sigma}_\tuple{n}
=\I\bigl(\tens C\cdot\vec k_\tuple{n}\bigr)\cdot\tilde{\vec u}_\tuple{n}
+\tilde{\tens\tau}_\tuple{n}.\end{split}\end{equation*}


The Cauchy stress tensor being symmetric, the first of the above equations also reads \(\vec k_\tuple{n}\cdot\tilde{\tens{\sigma}}_\tuple{n}=\vec 0\) and



\begin{equation*}
\begin{split}\tilde{\vec u}_\tuple{n}
=\I\bigl(\vec k_\tuple{n}\cdot\tens C\cdot\vec k_\tuple{n}\bigr)^{-1}
\cdot\tilde{\tens\tau}_\tuple{n}\cdot\vec k_\tuple{n}\end{split}\end{equation*}


which delivers the following expression for the Fourier modes of the strain field



\begin{equation*}
\begin{split}\tilde{\tens\varepsilon}_\tuple{n}
=-\tfrac12\bigl[\bigl(\vec k_\tuple{n}\cdot\tens C\cdot\vec k_\tuple{n}\bigr)^{-1}
\cdot\tilde{\tens\tau}_\tuple{n}\cdot\vec k_\tuple{n}\bigr]\otimes\vec k_\tuple{n}
-\tfrac12\vec k_\tuple{n}
\otimes\bigl[\bigl(\vec k_\tuple{n}\cdot\tens C\cdot\vec k_\tuple{n}\bigr)^{-1}
\cdot\tilde{\tens\tau}_\tuple{n}\cdot\vec k_\tuple{n}\bigr].\end{split}\end{equation*}


The above relation defines a linear mapping between \(\tilde{\tens\tau}_\tuple{n}\) and \(\tilde{\tens\varepsilon}_\tuple{n}\). For each Fourier mode \(\tuple{n}\), we therefore introduce the fourth-order tensor \(\tilde{\tens\Gamma}_\tuple{n}\) with major and minor symmetries, such that \(\tilde{\tens\varepsilon}_\tuple{n}=-\tilde{\tens\Gamma}_\tuple{n}\dbldot{\tilde{\tens\tau}}_\tuple{n}\). Our analysis shows that \(\tilde{\tens\Gamma}_\tuple{n}=\hat{\tens\Gamma}(\vec k_\tuple{n})\) where, for arbitrary wave-vector \(\vec k\), \(\hat{\tens\Gamma}(\vec k)\) is a fourth-order tensor with major and minor symmetries, such that



\begin{equation*}
\begin{split}\hat{\tens\Gamma}(\vec k)\dbldot\tilde{\tens\tau}
=\tfrac12\bigl[\bigl(\vec n\cdot\tens C\cdot\vec n\bigr)^{-1}
\cdot\tilde{\tens\tau}\cdot\vec n\bigr]\otimes\vec n
+\tfrac12\vec n\otimes\bigl[\bigl(\vec n\cdot\tens C\cdot\vec n\bigr)^{-1}
\cdot\tilde{\tens\tau}\cdot\vec n\bigr],\end{split}\end{equation*}


where \(\vec n=\vec k/\lVert\vec k\rVert\). The above equation defines \(\hat{\tens\Gamma}(\vec k)\) by how it operates on second-order, symmetric tensors. A closed-form expression of this tensor can be derived in the case of an isotropic material, for which



\begin{equation*}
\begin{split}\tens C=\lambda\tens I_2\otimes\tens I_2+2\mu\tens I_4,\end{split}\end{equation*}


where \(\tens I_2\) (resp. \(\tens I_4\)) is the second-order (resp. fourth-order) identity tensor, and \(\lambda\), \(\mu\) are the Lamé coefficients. Then



\begin{equation*}
\begin{split}\vec n\cdot\bigl(\tens I_2\otimes\tens I_2\bigr)\cdot\vec n=\vec n\otimes\vec n\end{split}\end{equation*}


then (recalling that \(\lVert\vec n\rVert=1\))



\begin{equation*}
\begin{split}\begin{aligned}
\vec n\cdot\tens I_4\cdot\vec n
&=\tfrac12 n_i\bigl(\delta_{ik}\delta_{jl}+\delta_{il}\delta_{jk}\bigr)n_l
\vec e_j\otimes\vec e_k
=\tfrac12\bigl(n_kn_j+n_in_i\delta_{jk}\bigr)\vec e_j\otimes\vec e_k\\
&=\tfrac12\bigl[\vec n\otimes\vec n+\bigl(\vec n\cdot\vec n\bigr)\tens I_2\bigr]
=\tfrac12\bigl(\vec n\otimes\vec n+\tens I_2\bigr)
=\vec n\otimes\vec n+\tfrac12\bigl(\tens I_2-\vec n\otimes\vec n\bigr)
\end{aligned}\end{split}\end{equation*}


and finally, we find the following expression of the “matrix-vector” product.



\hypertarget{17469303166468401856}{}


\subsection{The “matrix-vector” product for the Green operator of isotropic, linear elasticity}



\begin{equation*}
\begin{split}\vec n\cdot\tens C\cdot\vec n
=\bigl(\lambda+2\mu\bigr)\vec n\otimes\vec n
+\mu\bigl(\tens I_2-\vec n\otimes\vec n\bigr)
=2\mu\frac{1-\nu}{1-2\nu}\vec n\otimes\vec n
+\mu\bigl(\tens I_2-\vec n\otimes\vec n\bigr),\end{split}\end{equation*}


where \(\nu\) denotes the Poisson ratio. The above second-order tensor is easily inverted, since \(\vec n\otimes\vec n\) and \(\tens I_2-\vec n\otimes\vec n\) are two orthogonal projectors (in the sense of the “\(\dbldot\)” product)



\begin{equation*}
\begin{split}2\mu\bigl(\vec n\cdot\tens C\cdot\vec n\bigr)^{-1}
=\frac{1-2\nu}{1-\nu}\vec n\otimes\vec n+2\bigl(\tens I_2-\vec n\otimes\vec n\bigr)
=2\tens I_2-\frac{\vec n\otimes\vec n}{1-\nu},\end{split}\end{equation*}


from which it results that



\begin{equation*}
\begin{split}2\mu\bigl(\vec n\cdot\tens C\cdot\vec n\bigr)^{-1}
\cdot\tilde{\tens\tau}\cdot\vec n
=2\tilde{\tens\tau}\cdot\vec n
-\frac{\vec n\cdot\tilde{\tens\tau}\cdot\vec n}{1-\nu}\vec n\end{split}\end{equation*}


and we finally get



\hypertarget{7816488992572278472}{}


\subsection{The components of the Green operator of isotropic, linear elasticity}



\begin{equation*}
\begin{split}2\mu\hat{\tens \Gamma}(\vec k)\dbldot\tilde{\tens \tau}
=\bigl(\tilde{\tens \tau}\cdot\vec n\bigr)\otimes\vec n
+\vec n\otimes\bigl(\tilde{\tens \tau}\cdot\vec n\bigr)
-\frac{\vec n\cdot\tilde{\tens \tau}\cdot\vec n}{1-\nu}\vec n\otimes\vec n.\end{split}\end{equation*}


The components of the \(\hat{\tens \Gamma}\) tensor are then readily found



\begin{equation*}
\begin{split}\hat{\Gamma}_{ijkl}(\vec k)
=\frac{\delta_{ik}n_jn_l+\delta_{il}n_jn_k+\delta_{jk}n_in_l+\delta_{jl}n_in_k}{4\mu}
-\frac{n_in_jn_kn_l}{2\mu\bigl(1-\nu\bigr)},\end{split}\end{equation*}


which coincide with classical expressions (see e.g. \hyperlink{16288211201909122901}{Suquet, 1990}). Implementation of the above equation is cumbersome; it is only used for testing purposes. In Scapin, only the \texttt{matvec} product is required, and \hyperlink{7864328272685039520}{the “matrix-vector” product} was implemented.



\hypertarget{16861761838597320957}{}


\section{Hyperelasticity}



\hypertarget{1050018100428169985}{}


\chapter{Discrete Green operators}



In this chapter, we introduce various discretizations of the Green operator; we will adopt the vocabulary of linear elasticity, although the concepts apply to all the various physical models presented in Chap. “\hyperlink{11249657459891667799}{Continuous Green operators}”.



\hypertarget{5250349235821641230}{}


\section{On the discrete Fourier transform}



Let \(x=(x_{\tuple{p}})\) be a finite set of scalar values indexed by the \(d\)-tuple \(\tuple{p}=(p_1, \ldots, p_d)\) where \(0\leq p_i<N_i\) (\(N_i\) is the number of data points in the \(i\)-th direction). The discrete Fourier transform is a discrete set of scalar values \(\dft_{\tuple{n}}(x)\) indexed by the \(d\)-tuple \(\tuple{n}\in\integers^d\), defined as follows



\begin{equation*}
\begin{split}\dft_{\tuple{n}}(x)=\sum_{p_1=0}^{N_1-1}\cdots\sum_{p_d=0}^{N_d-1}
\exp\Bigl[-2\I\PI\Bigl(\frac{n_1p_1}{N_1}
+\cdots+\frac{n_dp_d}{N_d}\Bigr)\Bigr] x_{\tuple{p}}.\end{split}\end{equation*}


Note that in the above definition, no restrictions are applied to the multi-index \(\tuple{n}\). However, it can be verified that the above series of tensors is in fact \(\tuple{N}\)-periodic: \(\dft_{\tuple{n}+\tuple{N}}(x)=\dft_{\tuple{n}}(x)\), where \(\tuple{n}+\tuple{N}=(n_1+N_1, \ldots, n_d+N_d)\). Therefore, the \(\tuple{n}\)-index is effectively restricted to \(0\leq n_i<N_i\) as well. The most important results concerning the DFT are the \emph{inversion formula}



\begin{equation*}
\begin{split}x_{\tuple{p}}=\frac1{\lvert\tuple{N}\rvert}\sum_{n_1=0}^{N_1-1}\cdots
\sum_{n_d=0}^{N_d-1}\exp\Bigl[2\I\PI\Bigl(\frac{n_1p_1}{N_1}+\cdots+
\frac{n_dp_d}{N_d}\Bigr)\Bigr]\dft_{\tuple{n}}(x),\end{split}\end{equation*}


the \emph{Plancherel theorem}



\begin{equation*}
\begin{split}\sum_{p_1=1}^{N_1-1}\cdots\sum_{p_d=1}^{N_d-1}\conj(x_{\tuple p})y_{\tuple p}
=\frac1{\lvert\tuple N\rvert}\sum_{n_1=1}^{N_1-1}\cdots\sum_{n_d=1}^{N_d-1}
\conj[\dft_{\tuple n}(x)]\dft_{\tuple n}(y),\end{split}\end{equation*}


and the \emph{circular convolution theorem}



\begin{equation*}
\begin{split}\dft_{\tuple{n}}(x\ast y)=\dft_{\tuple{n}}(x)\dft_{\tuple{n}}(y),
\quad\text{where}\quad
(x\ast y)_{\tuple p}=\sum_{q_1=0}^{N_1-1}\cdots\sum_{q_d=0}^{N_d-1}
x_{\tuple{q}}y_{\tuple{p}-\tuple{q}}.\end{split}\end{equation*}


The DFT is readily extended to tensor data points. In the absence of ambiguity, the shorthand \(\hat{x}_{\tuple{n}}\) will be adopted for \(\dft_{\tuple{n}}(x)\).



To close this section, we observe that the DFT of a series of \emph{real} data points is a series of \emph{complex} data points. However, these complex values have the following property



\begin{equation*}
\begin{split}\dft_{\tuple{N}-\tuple{n}}(x)=\conj[\dft_{\tuple{n}}(x)].\end{split}\end{equation*}


The above condition is actually a \emph{necessary and sufficient} condition for the \(x_{\tuple{p}}\) to be real.



\hypertarget{2160060583997998814}{}


\subsection{The \(\fftfreq\) function}



For \(n, N\in\naturals\), \(0\leq n<N\), we introduce \(\fftfreq(n, N)\)



\begin{equation*}
\begin{split}\fftfreq(n, N)=
\begin{cases}
n & \text{if }2n<N,\\
n-N & \text{otherwise.}
\end{cases}\end{split}\end{equation*}


For \(n<0\) or \(n\geq N\), \(\fftfreq(n, N)\) is defined by \(N\)-periodicity. \(\fftfreq\) is very similar to the NumPy \href{https://numpy.org/doc/1.18/reference/generated/numpy.fft.fftfreq.html\#numpy.fft.fftfreq}{fftfreq} function. We have the important result (see proof in Sec. “\hyperlink{15063944043531008847}{Properties of the \(\fftfreq\) function}”.



\begin{equation*}
\begin{split}\fftfreq(N-n, N)=
\begin{cases}
\fftfreq(n) & \text{if }2n=N,\\
-\fftfreq(n) & \text{otherwise.}
\end{cases}\end{split}\end{equation*}


The \(\fftfreq\) function can be defined for \(d\)-tuples as well



\begin{equation*}
\begin{split}\tuple \fftfreq(\tuple n, \tuple N)
=\bigl(\fftfreq(n_1, N_1), \ldots, \fftfreq(n_d, N_d)\bigr)\end{split}\end{equation*}


and we have again



\begin{equation*}
\begin{split}\tuple \fftfreq(\tuple N-\tuple n, \tuple N)=-\tuple \fftfreq(\tuple n)\end{split}\end{equation*}


if none of the \(n_i\) is such that \(2n_i=N_i\).



\hypertarget{3470771881996884583}{}


\section{The approximation space}



In order to define a discrete Green operator, we need to introduce the approximation space for the stress-polarizations. We will consider here stress-polarizations that are constant over each cell of a regular grid of size \(N_1\times\cdots\times N_d\). The cells of this grid are



\begin{equation*}
\begin{split}\Omega_{\tuple{p}}^{\tuple{h}}=\{p_ih_i\leq x_i<\bigl(p_i+1\bigr)h_i, i=1,\ldots, d\},\end{split}\end{equation*}


where \(\tuple{p}=(p_1,\ldots,p_d)\in\cellindices\) denotes a \(d\)-tuple of integers, such that \(0\leq p_i<N_i\), \(i=1,\ldots, d\) and \(h_i=L_i/N_i\) is the cell-size in the \(i\)-th direction. Note that, in the above expression, no summation is implied by the repeated index \(i\). The total number of cells is \(\lvert N\rvert=N_1\cdots N_d\).



We consider discrete stress-polarizations \(\tens\tau^{\tuple{h}}\) that are constant over each cell of the grid: \(\tens\tau_{\tuple{p}}^{\tuple{h}}\) denotes the constant value of \(\tens\tau^{\tuple{h}}\) in cell \(\Omega_{\tuple{p}}^{\tuple{h}}\). The \(\tuple{h}\) superscript reminds that \(\tens\tau^{\tuple{h}}\) is a discrete approximation of the true stress-polarization \(\tens\tau\). We will call this approximation subspace: \(\tensors_2^{\tuple{h}}(\Omega)\).



As discussed \textbf{TODO: xref}, the discrete Green operator is defined as the restriction to this approximation space of the continuous Green operator, seen as a bilinear form, or an approximation of it. In other words, we want to propose an approximation of the quantity



\begin{equation*}
\begin{split}\langle\tens\varpi^{\tuple{h}}\dbldot\tens\Gamma(\tens\tau^{\tuple{h}})\rangle
\simeq\langle\tens\varpi^{\tuple{h}}\dbldot\tens\Gamma^{\tuple{h}}(\tens
\tau^{\tuple{h}})\rangle\quad\text{for all }\tens\tau^{\tuple{h}},
\tens\varpi^{\tuple{h}}\in\tensors_2^{\tuple{h}}(\Omega),\end{split}\end{equation*}


where \(\tens\Gamma^{\tuple{h}}\) is defined only over \(\tensors_2(\Omega)\). \(\tens\Gamma^{\tuple{h}}\) can therefore be seen as a linear mapping between the cell values \(\tens\tau_{\tuple{p}}^{\tuple{h}}\) of \(\tens\tau^{\tuple{p}}\) and the cell values of \(\tens\Gamma^{\tuple{h}}(\tens\tau^{\tuple{h}})\); \(\tens\Gamma^{\tuple{h}}\) is therefore a \emph{matrix}, and the above equation should be understood as



\begin{equation*}
\begin{split}\langle\tens\varpi^{\tuple{h}}\dbldot\tens\Gamma(\tens\tau^{\tuple{h}})\rangle
\simeq\frac1{\lvert\tuple{N}\rvert}\sum_{\tuple{p}, \tuple{q}\in\cellindices}
\tens\varpi_{\tuple{p}}^{\tuple{h}}\dbldot
\tens\Gamma_{\tuple{p}\tuple{q}}^{\tuple{h}}\dbldot\tens
\tau_{\tuple{q}}^{\tuple{h}}
\quad\text{for all}\quad
\tens\tau^{\tuple{h}},\tens\varpi^{\tuple{h}}\in\tensors_2^{\tuple{h}}(\Omega).\end{split}\end{equation*}


The continuous Green operator is translation invariant, and this property will of course be transferred to the “exact” discrete Green operator; we will in fact require \emph{all} dicretizations of the Green operator to have this property. In other words, \(\tens\Gamma_{\tuple{p}\tuple{q}}^{\tuple{h}}=\tens\Gamma_{\tuple p-\tuple q}^{\tuple{h}}\) and the above equation reads



\begin{equation*}
\begin{split}\langle\tens\varpi^{\tuple{h}}\dbldot\tens\Gamma(\tens\tau^{\tuple{h}})\rangle
\simeq\frac1{\lvert\tuple{N}\rvert}\sum_{\tuple{p}, \tuple{q}\in\cellindices}
\tens\varpi_{\tuple{p}}^{\tuple{h}}\dbldot
\tens\Gamma_{\tuple p-\tuple q}^{\tuple{h}}\dbldot\tens
\tau_{\tuple{q}}^{\tuple{h}}\quad\text{for all }\tens\tau^{\tuple{h}},\tens
\varpi^{\tuple{h}}\in\tensors_2^{\tuple{h}}(\Omega).\end{split}\end{equation*}


Note that \(\tens\Gamma_{\tuple{p}}^{\tuple{h}}\) is now indexed by \emph{one} index only and its DFT can be introduced unambiguously



\begin{equation*}
\begin{split}\begin{aligned}[b]
&\frac{1}{\lvert\tuple N\rvert}\sum_{\tuple{p}, \tuple{q}\in\cellindices}
\tens\varpi_{\tuple{p}}^{\tuple{h}}\dbldot\tens\Gamma_{\tuple p-\tuple q}^{\tuple h}
\dbldot\tens\tau_{\tuple{q}}^{\tuple{h}}\\
={}&\frac1{\lvert\tuple N\rvert^2}\sum_{\tuple p, \tuple q, \tuple n\in\cellindices}
\exp\Bigl[2\I\PI\sum_{j=1}^d\frac{n_j}{N_j}\bigl(p_j-q_j\bigr)\Bigr]
\tens\varpi_{\tuple{p}}^{\tuple{h}}\dbldot\hat{\tens\Gamma}_{\tuple n}^{\tuple h}
\dbldot\tens\tau_{\tuple q}^{\tuple h}\\
={}&\frac{1}{\lvert\tuple N\rvert^2}
\sum_{\tuple n\in\cellindices}\Bigl[\sum_{\tuple{p}\in\cellindices}
\exp\Bigl(2\I\PI\sum_{j=1}^d\frac{n_jp_j}{N_j}\Bigr)
\tens\varpi_{\tuple p}^{\tuple h}\Bigr]\dbldot\hat{\tens\Gamma}_{\tuple n}^{\tuple h}
\dbldot\Bigl[\sum_{\tuple q\in\cellindices}\exp\Bigl(-2\I\PI\sum_{j=1}^d
\frac{n_jq_j}{N_j}\tens\tau_{\tuple q}^{\tuple{h}}\Bigr)\Bigr].
\end{aligned}\end{split}\end{equation*}


Since \(\tens\varpi\) is real, we have \(\conj(\tens\varpi_{\tuple p}^h)=\tens\varpi_{\tuple p}^h\) and the first sum in square brackets reads



\begin{equation*}
\begin{split}\sum_{\tuple{p}\in\cellindices}\exp\Bigl(2\I\PI\sum_{j=1}^d\frac{n_jp_j}{N_j}\Bigr)
\tens\varpi_{\tuple p}^{\tuple h}=\conj\Bigl[\sum_{\tuple{p}\in\cellindices}
\exp\Bigl(-2\I\PI\sum_{j=1}^d\frac{n_jp_j}{N_j}\Bigr)
\tens\varpi_{\tuple p}^{\tuple h}\Bigr]
=\conj(\hat{\tens\varpi}_{\tuple n}^{\tuple h}),\end{split}\end{equation*}


while the second sum in square brackets reduces to \(\hat{\tens\tau}_{\tuple n}^{\tuple h}\). Gathering the above results, we find



\begin{equation*}
\begin{split}\frac{1}{\lvert\tuple N\rvert}\sum_{\tuple p, \tuple q\in\cellindices}
\tens\varpi_{\tuple p}^{\tuple h}\dbldot\tens\Gamma_{\tuple p-\tuple q}^{\tuple h}
\dbldot\tens\tau_{\tuple q}^{\tuple h}=\frac1{\lvert\tuple N\rvert^2}
\sum_{\tuple n\in\cellindices}\conj(\hat{\tens\varpi}_{\tuple n}^{\tuple h})
\dbldot\hat{\tens\Gamma}_{\tuple n}^{\tuple h}\dbldot
\hat{\tens\tau}_{\tuple n}^{\tuple h}.\end{split}\end{equation*}


The above equation can be understood as follows. \(\tens\Gamma^{\tuple h}\) is a linear operator that maps the cell-wise constant field \(\tens\tau^{\tuple h}\) to the cell-wise constant field \(\tens\eta^{\tuple h}\), the cell-values of which are given by their DFT



\begin{equation*}
\begin{split}\tens{\eta}_{\tuple p}^{\tuple h}=
\dft^{-1}_{\tuple p}(\hat{\tens\eta}_{\bullet}^{\tuple h}),\quad\text{with}
\quad\hat{\tens\eta}_{\tuple n}^{\tuple h}
=\hat{\tens\Gamma}_{\tuple n}^{\tuple h}\dbldot
\hat{\tens\tau}_{\tuple n}^{\tuple h}.\end{split}\end{equation*}


Then, from the Plancherel theorem



\begin{equation*}
\begin{split}\frac1{\lvert\tuple N\rvert^2}
\sum_{\tuple n\in\cellindices}\conj(\hat{\tens\varpi}_{\tuple n}^{\tuple h})
\dbldot\hat{\tens\Gamma}_{\tuple n}^{\tuple h}\dbldot
\hat{\tens\tau}_{\tuple n}^{\tuple h}=\frac1{\lvert\tuple N\rvert^2}
\sum_{\tuple n\in\cellindices}\conj(\hat{\tens\varpi}_{\tuple n}^{\tuple h})
\dbldot\hat{\tens\eta}_{\tuple n}^{\tuple h}=\frac1{\lvert\tuple N\rvert}
\sum_{\tuple p\in\cellindices}\tens\varpi_{\tuple p}^{\tuple h}
\dbldot\tens\eta_{\tuple n}^{\tuple h}\end{split}\end{equation*}


and the last sum can be seen as the volume average \(\langle\tens\varpi^{\tuple h}\dbldot\tens\eta^{\tuple h}\rangle\). Remembering that this expression was proposed as an approximation of \(\langle\tens\varpi^{\tuple h}\dbldot\tens\Gamma(\tens\tau^{\tuple h})\rangle\), we finally find



\begin{equation*}
\begin{split}\langle\tens\varpi^{\tuple h}\dbldot\tens\Gamma(\tens\tau^{\tuple h})\rangle
\simeq\langle\tens\varpi^{\tuple h}\dbldot\tens\eta^{\tuple h}\rangle
=\langle\tens\varpi^{\tuple h}\dbldot\tens\Gamma^{\tuple h}(\tens\tau^{\tuple h})\rangle
\quad\text{for all }\tens\varpi^{\tuple h}\in\tensors_2^{\tuple h}(\Omega),\end{split}\end{equation*}


from which we find



\begin{equation*}
\begin{split}\tens\Gamma(\tens\tau^{\tuple h})\simeq\tens\Gamma^{\tuple h}
(\tens\tau^{\tuple h}).\end{split}\end{equation*}


The discrete Green operator, which was first introduced as an approximation of the continuous Green operator, seen as a bilinear form, can also be understood as an approximation of the continuous Green operator, seen as a linear mapping. This latter point of view will become extremely efficient when it comes to discretizing the Lippmann–Schwinger equation.



It results from the above developments that an explicit expression of the discrete Green operator as a (gigantic) matrix is never needed. Instead, the matrix-vector product \(\tens\tau^{\tuple h}\mapsto\tens\Gamma^{\tuple h}(\tens\tau^{\tuple h})\) is implemented in a matrix-free fashion as follows



\begin{itemize}
\item[1. ] Given \(\tens\tau^{\tuple h}\in\tensors_2^{\tuple h}(\Omega)\), compute the discrete Fourier transform \(\hat{\tens\tau}_{\tuple n}^{\tuple h}\) of its cell-values:\(\hat{\tens\tau}_{\tuple n}^{\tuple h}=\dft_{\tuple n}(\tens\tau_\bullet^{\tuple h})\),


\item[2. ] for each discrete frequency, compute \(\hat{\tens\eta}_{\tuple n}^{\tuple h}=\hat{\tens\Gamma}_{\tuple n}^{\tuple h}\dbldot\hat{\tens\tau}_{\tuple n}^{\tuple h}\),


\item[3. ] compute the inverse discrete Fourier transform \(\tens\eta_{\tuple{p}}^{\tuple h}\) of \(\hat{\tens\eta}_{\tuple{n}}^{\tuple h}\),

\end{itemize}


discrete Fourier transforms being computed in steps 1 and 3 by means of the FFT.



\hypertarget{8407349361351071864}{}


\subsection{Condition for the discrete Green operator to map real fields onto real fields}



In the remainder of this chapter, we propose various discretizations of the Green operator. Before we proceed, though, it should be emphasized that the discrete Green operator must map a \emph{real} field onto a \emph{real} field. In other words, we must have \(\hat{\tens\eta}_{\tuple N-\tuple n}^{\tuple h}=\conj(\hat{\tens\eta}_{\tuple n}^{\tuple h})\) for all \(\tuple n\). Since \(\hat{\tens\eta}_{\tuple n}^{\tuple h} =\hat{\tens\Gamma}_{\tuple n}^{\tuple h}\dbldot\hat{\tens\tau}_{\tuple n}^{\tuple h}\) and \(\hat{\tens\tau}_{\tuple n}\) already satisfies this condition (it is the DFT of a \emph{real} field), we find the following requirement.



Any discrete operator that will be considered below must ensure that



\begin{equation*}
\begin{split}\hat{\tens\Gamma}_{\tuple N-\tuple n}^{\tuple h}
=\conj(\hat{\tens\Gamma}_{\tuple n}^{\tuple h})
\quad\text{for all}\quad\tuple n\in\cellindices.\end{split}\end{equation*}


\hypertarget{11094457950407373392}{}


\section{Discretizations of the Green operator}



\hypertarget{11847327408909839613}{}


\subsection{The discretization of Brisard and Dormieux}



It was proved by Brisard and Dormieux \hyperlink{17188227464826042090}{(2010)} that, for all \(\tens\tau^{\tuple h}, \tens\varpi^{\tuple h}\in\tensors_2^{\tuple h}(\Omega)\)



\begin{equation*}
\begin{split}\langle\tens\varpi^h\dbldot\tens\Gamma(\tens\tau^h)\rangle=\frac1{N^2}
\sum_{n_1=0}^{N_1-1}\cdots\sum_{n_d=0}^{N_d-1}\conj(\hat{\tens\varpi}_
{\tuple{n}}^h)\dbldot\hat{\tens\Gamma}_n^{h, \mathrm{BD10}}\dbldot
\hat{\tens\tau}_{\tuple{n}}^h,\end{split}\end{equation*}


where



\begin{equation*}
\begin{split}\hat{\tens\Gamma}_{\tuple n}^{\tuple h, \mathrm{BD10}}
=\sum_{\tuple m\in\integers^d}\bigl[F(\vec\alpha_{\tuple n+\tuple m\tuple N})
\bigr]^2\hat{\tens\Gamma}(\vec k_{\tuple n+\tuple m\tuple N}),\end{split}\end{equation*}


where \(\tuple{n+mN}\) denotes the \(d\)-tuple: \(\tuple n+\tuple m\tuple N=(n_1+m_1N_1, \ldots, n_d+m_dN_d)\), while \(\alpha_{\tuple{n}}\) is the following dimensionless vector



\begin{equation*}
\begin{split}\vec\alpha_{\tuple{n}}
=\frac{2\pi h_1n_1}{L_1}\vec e_1+\cdots+\frac{2\pi h_dn_d}{L_d}\vec e_d,\end{split}\end{equation*}


finally, \(F\) is the tensor product of sine cardinal functions



\begin{equation*}
\begin{split}F(\vec\alpha)=\sinc\frac{\alpha_1}2\cdots\sinc\frac{\alpha_d}2.\end{split}\end{equation*}


Note that the above equations deliver the \emph{exact} cell-averages of the Green operator, applied exactly to any \emph{cell-wise constant} polarization field. Unfortunately, the series that defines the discrete Green operator can in general not be evaluated, owing to very slow convergence. Therefore, this discrete Green operator is unpractical, and is recalled here only for pedagocial reasons.



\hypertarget{4061870237023517960}{}


\subsection{The discretization of Moulinec and Suquet}



This is probably the most simple discretization, introduced first by Moulinec and Suquet (\hyperlink{1963195518433610412}{1994}, \hyperlink{7941652814461145344}{1998}). Only the lowest (positive and negative) frequencies are kept



\begin{equation*}
\begin{split}\hat{\tens\Gamma}_{\tuple n}^{\tuple h, \mathrm{MS94}}=\hat{\tens\Gamma}
(\vec k_{\tuple Z(\tuple n, \tuple N)}).\end{split}\end{equation*}


We must check that the “\hyperlink{4640340478983239538}{Condition for the discrete Green operator to map real fields onto real fields}” is satisfied. Using the “\hyperlink{15063944043531008847}{Properties of the \(\fftfreq\) function}” of the \(\fftfreq\) function and assuming first that none of the \(n_i\) is such that \(2n_i=N_i\)



\begin{equation*}
\begin{split}\hat{\tens\Gamma}_{\tuple N-\tuple n}^{\tuple h, \mathrm{MS94}}
=\hat{\tens\Gamma}(\vec k_{\tuple Z(\tuple N-\tuple n, \tuple N)})
=\hat{\tens\Gamma}(\vec k_{-\tuple Z(\tuple n, \tuple N)})
=\hat{\tens\Gamma}(-\vec k_{\tuple Z(\tuple n, \tuple N)})\end{split}\end{equation*}


All Green operators presented in Chap. “\hyperlink{11249657459891667799}{Continuous Green operators}” are such that \(\hat{\tens\Gamma}(-\vec k)=\hat{\tens\Gamma}(\vec k)\), therefore



\begin{equation*}
\begin{split}\hat{\tens\Gamma}_{\tuple N-\tuple n}^{\tuple h, \mathrm{MS94}}
=\hat{\tens\Gamma}(\vec k_{\tuple Z(\tuple n, \tuple N)})
=\hat{\tens\Gamma}_{\tuple n}^{\tuple h, \mathrm{MS94}}\end{split}\end{equation*}


and the property is verified. Conversely, if all the \(n_i\) are such that \(2n_i=N_i\), then



\begin{equation*}
\begin{split}\hat{\tens\Gamma}_{\tuple N-\tuple n}^{\tuple h, \mathrm{MS94}}
=\hat{\tens\Gamma}(\vec k_{\tuple Z(\tuple N-\tuple n, \tuple N)})
=\hat{\tens\Gamma}(\vec k_{\tuple Z(\tuple n, \tuple N)})
=\hat{\tens\Gamma}_{\tuple n}^{\tuple h, \mathrm{MS94}}.\end{split}\end{equation*}


More problematic is the case when a few, but not all, \(n_i\) are such that \(2n_i=N_i\). Then the property does not hold for such frequencies. Moulinec and Suquet (\hyperlink{7941652814461145344}{1998}) use a specific treatment for such cases



\begin{equation*}
\begin{split}\hat{\tens\Gamma}(\vec k_{\tuple n})=\tens C^{-1},\end{split}\end{equation*}


if one of the \(n_i\) is such that \(2n_i=N_i\). This is implemented in \texttt{Scapin}. Note that such cases occur only for even-sized grids.



\hypertarget{17831501573587717000}{}


\chapter{On the d-dimensional brick element}



In this chapter, we formulate the linear brick finite element for conductivity and linear elasticity in \(d\) dimensions (\(d\in\{2, 3\}\)).



\begin{quote}
\textbf{Note}

In the remainder of this chapter, \(i\), \(j\), \(h\) and \(k\) denote scalar indices that refer to tensor components, and span \(\{1, 2, \cdots d\}\); \(\tuple{m}\), \(\tuple{n}\) are \(d\)-dimensional multi-indices that refer to nodes. Depending on the context (local, element operators or global, assembled operators) they span \(\{1, 2\}^d\) or the whole grid \(\cellindices\). We adopt Einstein{\textquotesingle}s convention for indices of both types.

\end{quote}


\hypertarget{10162273478024495150}{}


\section{Geometry of the reference brick element}



The reference element being centered at the origin, it occupies the following domain



\begin{equation*}
\begin{split}\Omega^\element=\biggl(-\frac{h_1}2, \frac{h_1}2\biggr)\times\cdots
\times\biggl(-\frac{h_d}2, \frac{h_d}2\biggr),\end{split}\end{equation*}


where \(h_i\) is the size of the element along axis \(i\) (\(i=1, \dots, d\)). The nodes of the element are indexed by \(\tuple{n}\in\{1, 2\}^d\), such that the coordinates of node \(\tuple{n}\) read



\begin{equation*}
\begin{split}x_{\tuple{n}i}=(-1)^{n_i}\frac{h_i}2.\end{split}\end{equation*}


For integration purposes, it is useful to introduce the \emph{reduced coordinates} \(\xi_i=2x_i/h_i\)



\begin{equation*}
\begin{split}\int_{\Omega^\element}f(x_1, \ldots, x_d)\,\D x_1\cdots\D x_d
=\frac{h_1\cdots h_d}{2^d}\int_{(-1, 1)^d}f(h_1\xi_1/2, \ldots, h_d\xi_d/2)
\,\D\xi_1\cdots\D\xi_d\end{split}\end{equation*}


and it is observed that the reduced coordinates of the node \(\tuple{n}\) are \(\xi_i=(-1)^{n_i}\).



\hypertarget{17125853350239781481}{}


\section{Shape functions}



The shape function associated with node \(\tuple{n}\) is \(N^\element_{\tuple{n}}(\vec x)\), wich is defined below as a function of the reduced coordinates of the observation point



\begin{equation*}
\begin{split}N^\element_{\tuple{n}}(\vec x)=\frac1{2^d}\prod_{i=1}^d\bigl[1+(-1)^{n_i}\xi_i\bigr]\end{split}\end{equation*}


and we have as expected \(N^\element_{\tuple{m}}(\vec x_{\tuple n}) = \delta_{\tuple{m}\tuple{n}}\). The nodal values of \(f\) being \(f_\tuple{n}\), we have the interpolation formula: \(f(\vec x)=N^\element_\tuple{n}(\vec x)f_\tuple{n}\).



\hypertarget{2492262002713312068}{}


\section{Gradient and strain-displacement operators}



We consider a scalar interpolated field, \(f(\vec x)=N^\element_\tuple{n}(\vec x)f_\tuple{n}\). The components of its gradient are given by the following expression



\begin{equation*}
\begin{split}\bigl(f\otimes\nabla\bigr)_i=\partial_i f=D_{i\tuple{n}}^\element f_\tuple{n},\end{split}\end{equation*}


where the observation point \(\vec x\) has been dropped, and \(D_{i\tuple{n}}^\element\) denotes the element gradient operator



\begin{equation*}
\begin{split}D_{i\tuple{n}}^\element=\partial_iN_\tuple{n}
=\frac{(-1)^{n_i}}{2^{d-1}h_i}
\prod_{j\neq i}\bigl[1+(-1)^{n_j}\xi_j\bigr].\end{split}\end{equation*}


It will be useful to introduce the element average of this operator, \(\overline{D}_{i\tuple{n}}^\element\)



\begin{equation*}
\begin{split}\overline{D}_{i\tuple{n}}^\element=\frac 1{h_1\cdots h_d}
\int_{\Omega^\element}D^\element_{i\tuple{n}}(\vec x)\,\D x_1\cdots\D x_d\end{split}\end{equation*}


and, observing that each factor in square brackets in the above expression of \(D^\element_{i\tuple{n}}\) varies linearly between 0 et 2 (and therefore averages to 1), we find



\begin{equation*}
\begin{split}\overline{D}^\element_{i\tuple{n}}=\frac{(-1)^{n_i}}{2^{d-1}h_i}.\end{split}\end{equation*}


We now consider the interpolated displacement field, \(\vec u = N^\element_\tuple{n} \vec u_\tuple{n}\). The components of the associated strain, \(\tens\varepsilon=\vec u\symotimes\nabla\) are



\begin{equation*}
\begin{split}\varepsilon_{ij}=\tfrac12\bigl(\partial_iu_j+\partial_ju_i\bigr)
=\tfrac12\bigl(D^\element_{i\tuple{n}}u_{\tuple{n}j}
+D^\element_{j\tuple{n}}u_{\tuple{n}i}\bigr),\end{split}\end{equation*}


where \(u_{\tuple{n}i}\) denotes the \(i\)-th component of the nodal value of \(\vec u\) at node \(\tuple{n}\), \(u_{\tuple{n}i}=\vec u(\vec x_\tuple{n})\cdot\vec e_i\). Rearranging the above expression, we find \(\varepsilon_{ij}=B^\element_{ij\tuple{n}k}u_{\tuple{n}k}^\element\), where



\begin{equation*}
\begin{split}B_{ij\tuple{n}k}^\element
=\tfrac12\bigl(D^\element_{i\tuple{n}}\delta_{jk}
+D^\element_{j\tuple{n}}\delta_{ik}\bigr)\end{split}\end{equation*}


is the element strain-displacement operator, with volume average



\begin{equation*}
\begin{split}\overline{B}_{ij\tuple{n}k}^\element
=\tfrac12\bigl(\overline{D}^\element_{i\tuple{n}}\delta_{jk}
+\overline{D}^\element_{j\tuple{n}}\delta_{ik}\bigr)
=\frac1{2^d}\Bigl[\frac{(-1)^{n_i}\delta_{jk}}{h_i}
+\frac{(-1)^{n_j}\delta_{ik}}{h_j}\Bigr].\end{split}\end{equation*}


It is obverved that the \emph{trace} of the strain tensor is given by the following expression



\begin{equation*}
\begin{split}\varepsilon_{ii}=D_{i\tuple{n}}u_{\tuple{n}i}^\element\end{split}\end{equation*}


\hypertarget{15811403169534872275}{}


\section{Element stiffness operator}



\hypertarget{6549129120501784746}{}


\subsection{Conductivity}



\hypertarget{5017937499254797989}{}


\subsection{Linear elasticity}



Assuming the material to be linearly elastic, with Lamé coefficients \(\lambda\) and \(\mu\), we have the following expression of the interpolated stresses as a function of the nodal displacements \(u_{\tuple{m}i}\)



\begin{equation*}
\begin{split}\sigma_{hk}=\bigl(\lambda D^\element_{i\tuple{m}}\delta_{hk}
+2\mu B^\element_{hk\tuple{m}i}\bigr)u_{\tuple{m}i}\end{split}\end{equation*}


and the volume density of elastic energy reads



\begin{equation*}
\begin{split}\tfrac12\sigma_{hk}\varepsilon_{hk}
=\bigl(\lambda D^\element_{i\tuple{m}}\delta_{hk}
+2\mu B^\element_{hk\tuple{m}i}\bigr)B_{hk\tuple{n}j}^\element
u_{\tuple{m}i}u_{\tuple{n}j}.\end{split}\end{equation*}


Upon expansion and integration, we get the expression of the elastic energy within the element



\begin{equation*}
\begin{split}U^\element
=\int_{\Omega^\element}\tfrac12\sigma_{hk}\varepsilon_{hk}\,\D x_1\cdots\D x_d
=\tfrac12K^\element_{\tuple{m}i\tuple{n}j}u_{\tuple{m}i}u_{\tuple{n}j},\end{split}\end{equation*}


where the element stiffness operator reads



\begin{equation*}
\begin{split}K^\element_{\tuple{m}i\tuple{n}j}
=\lambda\,K^{\element, \lambda}_{\tuple{m}i\tuple{n}j}
+\mu\,K^{\element, \mu}_{\tuple{m}i\tuple{n}j},\end{split}\end{equation*}


with



\begin{equation*}
\begin{split}K^{\element, \lambda}_{\tuple{m}i\tuple{n}j}
=\int_{\Omega^\element}D^\element_{i\tuple{m}}D^\element_{j\tuple{n}}
\quad\text{and}\quad
K^{\element, \mu}_{\tuple{m}i\tuple{n}j}
=\int_{\Omega^\element} 2 B^\element_{hk\tuple{m}i}B^\element_{hk\tuple{n}j}.\end{split}\end{equation*}


\hypertarget{2489772129592300935}{}


\chapter{Appendix}



\hypertarget{18289269374885937903}{}


\section{Properties of the \(\fftfreq\) function}



In this paragraph, we prove the following property of the \(\fftfreq\) function



\begin{equation*}
\begin{split}\fftfreq(N-n, N)=
\begin{cases}
\fftfreq(n) & \text{if }2n=N,\\
-\fftfreq(n) & \text{otherwise.}
\end{cases}\end{split}\end{equation*}


Several cases must be considered.



\begin{itemize}
\item[1. ] If \(n=0\)

\begin{equation*}
\begin{split}\fftfreq(N-n, N)=\fftfreq(N, N)=\fftfreq(0, N)=0=-\fftfreq(n, N)\end{split}\end{equation*}

\item[2. ] If \(N\) is even, \(N=2M\)

\begin{itemize}
\item[1. ] If \(0<n<M\)

\begin{equation*}
\begin{split}\begin{gather*}
2n<N\quad\Rightarrow\quad \fftfreq(n, N)=n,\\
\begin{aligned}
M<N-n&\quad\Rightarrow\quad N<2\bigl(N-n\bigr)\\
&\quad\Rightarrow\quad \fftfreq(N-n, N)=N-n-N=-n.
\end{aligned}
\end{gather*}\end{split}\end{equation*}

\item[2. ] If \(n=M\)

\begin{equation*}
\begin{split}\begin{gather*}
2n=N\quad\Rightarrow\quad \fftfreq(n, N)=n-N=-M,\\
2(N-n)=2M=N\quad\Rightarrow\quad \fftfreq(N-n, N)=-M.
\end{gather*}\end{split}\end{equation*}

\item[3. ] If \(M<n<N\)

\begin{equation*}
\begin{split}\begin{gather*}
N<2n\quad\Rightarrow\quad \fftfreq(n, N)=n-N,\\
\begin{aligned}
N-n<M&\quad\Rightarrow\quad 2\bigl(N-n\bigr)<N\\
&\quad\Rightarrow\quad \fftfreq(N-n, N)=N-n.
\end{aligned}
\end{gather*}\end{split}\end{equation*}
\end{itemize}

\item[3. ] If \(N\) is odd, \(N=2M+1\)

\begin{itemize}
\item[1. ] If \(0<n\leq M\)

\begin{equation*}
\begin{split}\begin{gather*}
2n\leq 2M<N\quad\Rightarrow\quad \fftfreq(n, N)=n,\\
\begin{aligned}
M+1\leq N-n&\quad\Rightarrow\quad N<2\bigl(N-n\bigr)\\
&\quad\Rightarrow\quad \fftfreq(N-n, N)=N-n-N=-n.
\end{aligned}
\end{gather*}\end{split}\end{equation*}

\item[2. ] If \(M+1\leq n<N\)

\begin{equation*}
\begin{split}\begin{gather*}
N+1\leq 2n\quad\Rightarrow\quad \fftfreq(n, N)=n-N\\
\begin{aligned}
N-n\leq N-M-1=M&\quad\Rightarrow\quad 2\bigl(N-n\bigr)\leq 2M<N\\
&\quad\Rightarrow\quad \fftfreq(N-n, N)=N-n
\end{aligned}
\end{gather*}\end{split}\end{equation*}
\end{itemize}
\end{itemize}


\hypertarget{7261367053393854382}{}


\chapter{Bibliography}



\hypertarget{16002782079292950992}{}


\section{Brisard, S., \& Dormieux, L. (2010)}



FFT-based methods for the mechanics of composites: A general variational framework. \emph{Computational Materials Science}, \emph{49}(3), 663–671. \href{https://doi.org/10.1016/j.commatsci.2010.06.009}{DOI:10.1016/j.commatsci.2010.06.009}



\hypertarget{17627112247925688930}{}


\section{Brisard, S., \& Dormieux, L. (2012)}



Combining Galerkin approximation techniques with the principle of Hashin and Shtrikman to derive a new FFT-based numerical method for the homogenization of composites. \emph{Computer Methods in Applied Mechanics and Engineering}, \emph{217–220}, 197–212. \href{https://doi.org/10.1016/j.cma.2012.01.003}{DOI:10.1016/j.cma.2012.01.003}



\hypertarget{8653883748112457965}{}


\section{Moulinec, H., \& Suquet, P. (1994)}



A fast numerical method for computing the linear and nonlinear mechanical properties of composites. \emph{Comptes Rendus de l’Académie Des Sciences. Série II, Mécanique, Physique, Chimie, Astronomie}, \emph{318}(11), 1417–1423.



\hypertarget{9534833447558490031}{}


\section{Moulinec, H., \& Suquet, P. (1998)}



A numerical method for computing the overall response of nonlinear composites with complex microstructure. \emph{Computer Methods in Applied Mechanics and Engineering}, \emph{157}(1–2), 69–94. \href{https://doi.org/10.1016/S0045-7825(97)00218-1}{DOI:10.1016/S0045-7825(97)00218-1}



\hypertarget{15745057744408808743}{}


\section{Suquet, P. (1990)}



A simplified method for the prediction of homogenized elastic properties of composites with a periodic structure. \emph{Comptes-Rendus de l’Académie Des Sciences Série II}, \emph{311}(7), 769–774.



\hypertarget{10620780704054356859}{}


\section{Wicht, D., Schneider, M., \& Böhlke, T. (2020)}



On Quasi-Newton methods in fast Fourier transform-based micromechanics. \emph{International Journal for Numerical Methods in Engineering}, \emph{121}(8), 1665–1694. \href{https://doi.org/10.1002/nme.6283}{DOI:10.1002/nme.6283}



\part{Library}

\hypertarget{8158906180689097333}{} 
\hyperlink{8158906180689097333}{\texttt{Scapin.Hooke}}  -- {Type.}

\begin{adjustwidth}{2em}{0pt}

Isotropic, linear elastic material.


\begin{lstlisting}
Hooke{T, DIM}(μ::T, ν::T)
\end{lstlisting}

Create a new instance with shear modulus \texttt{μ} and Poisson ratio \texttt{ν}.

\begin{quote}
\textbf{Material stability}

Material stability requires that \texttt{μ > 0} and \texttt{-1 < ν < 1/2}; these conditions are \emph{not} enforced here. In other words, \emph{unstable} materials \emph{can} be defined.

\end{quote}
\begin{quote}
\textbf{Plane stresses vs. plane strains}

In the current implementation, \texttt{DIM = 2} refers to plane strain elasticity. For plane stresses, the \emph{true} Poisson ratio \texttt{ν} should be replaced with the \emph{fictitious} ratio \texttt{ν̃ = ν / (1 + ν)}.

\end{quote}


\href{https://github.com/sbrisard/Scapin.jl/blob/ea3f90c60e90cdf214e41b1314a8ee608e0b8d10/src/hooke.jl#L1-L18}{\texttt{source}}


\end{adjustwidth}
\hypertarget{11552838231886230083}{} 
\hyperlink{11552838231886230083}{\texttt{Scapin.avg\_gradient\_operator}}  -- {Method.}

\begin{adjustwidth}{2em}{0pt}


\begin{minted}{julia}
avg_gradient_operator(h)
\end{minted}

Return the cell average of the gradient operator.



\href{https://github.com/sbrisard/Scapin.jl/blob/ea3f90c60e90cdf214e41b1314a8ee608e0b8d10/src/bri17.jl#L191-L195}{\texttt{source}}


\end{adjustwidth}
\hypertarget{10684753069292900654}{} 
\hyperlink{10684753069292900654}{\texttt{Scapin.avg\_strain\_displacement\_operator}}  -- {Method.}

\begin{adjustwidth}{2em}{0pt}


\begin{minted}{julia}
avg_strain_displacement_operator(h)
\end{minted}

Return the cell average of the strain-displacement operator.



\href{https://github.com/sbrisard/Scapin.jl/blob/ea3f90c60e90cdf214e41b1314a8ee608e0b8d10/src/bri17.jl#L235-L239}{\texttt{source}}


\end{adjustwidth}
\hypertarget{1262405759471151845}{} 
\hyperlink{1262405759471151845}{\texttt{Scapin.bulk\_modulus}}  -- {Method.}

\begin{adjustwidth}{2em}{0pt}


\begin{minted}{julia}
bulk_modulus(C::Hooke)
\end{minted}

Return the bulk modulus \texttt{κ} for the specified Hooke material.

For plane strain elasticity


\begin{lstlisting}
κ = μ / (1 - 2ν),
\end{lstlisting}

and, for 3D elasticity


\begin{lstlisting}
κ = 2/3 μ (1 + ν) / (1 - 2ν).
\end{lstlisting}



\href{https://github.com/sbrisard/Scapin.jl/blob/ea3f90c60e90cdf214e41b1314a8ee608e0b8d10/src/hooke.jl#L51-L67}{\texttt{source}}


\end{adjustwidth}
\hypertarget{7598511707942506255}{} 
\hyperlink{7598511707942506255}{\texttt{Scapin.cell\_vertices}}  -- {Method.}

\begin{adjustwidth}{2em}{0pt}


\begin{minted}{julia}
cell_vertices(cell, linear)
\end{minted}

Return the vertices of the \texttt{cell} (specified as a multi-index) as an array of indices. The grid size is defined by \texttt{linear} (an instance of \texttt{LinearIndices}).



\href{https://github.com/sbrisard/Scapin.jl/blob/ea3f90c60e90cdf214e41b1314a8ee608e0b8d10/src/bri17.jl#L282-L288}{\texttt{source}}


\end{adjustwidth}
\hypertarget{6536319419682078378}{} 
\hyperlink{6536319419682078378}{\texttt{Scapin.element\_nodes}}  -- {Method.}

\begin{adjustwidth}{2em}{0pt}


\begin{minted}{julia}
element_nodes(d)
\end{minted}

Return the multi-indices of the \texttt{d}-dimensional brick element.

This function returns an object \texttt{𝔑} of type \texttt{CartesianIndices}. An element \texttt{n ∈ 𝔑} represents the node with coordinates \texttt{(x[1], …, x[d])}


\begin{lstlisting}
x[i] = (-1)^n[i] * h[i] / 2,    i = 1, …, d
\end{lstlisting}

See “\hyperlink{15016019730971184536}{Geometry of the reference brick element}” in the docs.



\href{https://github.com/sbrisard/Scapin.jl/blob/ea3f90c60e90cdf214e41b1314a8ee608e0b8d10/src/bri17.jl#L111-L124}{\texttt{source}}


\end{adjustwidth}
\hypertarget{11073413609374358964}{} 
\hyperlink{11073413609374358964}{\texttt{Scapin.gradient\_operator}}  -- {Method.}

\begin{adjustwidth}{2em}{0pt}


\begin{minted}{julia}
gradient_operator(x, h)
\end{minted}

Return the gradient operator at the specified point.

This function returns a \texttt{(d+1)} dimensional array \texttt{D} of size \texttt{(d, 2, 2, …)}. If \texttt{n} is the multi-index of the node, and \texttt{i} is the index of a component, then \texttt{D[i, n]} is the partial derivative of \texttt{N[n]} w.r.t. \texttt{x[i]}, evaluated at \texttt{x}.

\texttt{h} is the size of the brick element.

See “\hyperlink{15016019730971184536}{Geometry of the reference brick element}”.



\href{https://github.com/sbrisard/Scapin.jl/blob/ea3f90c60e90cdf214e41b1314a8ee608e0b8d10/src/bri17.jl#L166-L178}{\texttt{source}}


\end{adjustwidth}
\hypertarget{919789657356107432}{} 
\hyperlink{919789657356107432}{\texttt{Scapin.integrate}}  -- {Method.}

\begin{adjustwidth}{2em}{0pt}


\begin{minted}{julia}
integrate(f, h)
\end{minted}

Return the \texttt{N}-dimensional integral of \texttt{f} over \texttt{(0, h[1]) × (0, h[2]) × … × (0, h[N])}.

Uses 2-point Gauss-Legendre integration (tensorized over the \texttt{N} dimensions). \texttt{f} must take a 1-dimensional array of size \texttt{N} as unique input. If \texttt{avg} is \texttt{true}, the function returns the \texttt{N}-dimensional average.



\href{https://github.com/sbrisard/Scapin.jl/blob/ea3f90c60e90cdf214e41b1314a8ee608e0b8d10/src/bri17.jl#L128-L136}{\texttt{source}}


\end{adjustwidth}
\hypertarget{6415299341137554776}{} 
\hyperlink{6415299341137554776}{\texttt{Scapin.modal\_stiffness!}}  -- {Method.}

\begin{adjustwidth}{2em}{0pt}


\begin{minted}{julia}
modal_stiffness!(K, k, N, h, C)
\end{minted}

Compute the modal stiffness matrix \texttt{K} in place, for the spatial frequency \texttt{k}. The grid is defined by its size, \texttt{N} and spacing, \texttt{h}; \texttt{C} is the material.



\href{https://github.com/sbrisard/Scapin.jl/blob/ea3f90c60e90cdf214e41b1314a8ee608e0b8d10/src/bri17.jl#L45-L51}{\texttt{source}}


\end{adjustwidth}
\hypertarget{13564334315567723126}{} 
\hyperlink{13564334315567723126}{\texttt{Scapin.modal\_stiffness}}  -- {Method.}

\begin{adjustwidth}{2em}{0pt}


\begin{minted}{julia}
modal_stiffness!(K, k, N, h, C)
\end{minted}

Compute the modal stiffness matrix \texttt{K}, for the spatial frequency \texttt{k}. The grid is defined by its size, \texttt{N} and spacing, \texttt{h}; \texttt{C} is the material.



\href{https://github.com/sbrisard/Scapin.jl/blob/ea3f90c60e90cdf214e41b1314a8ee608e0b8d10/src/bri17.jl#L95-L101}{\texttt{source}}


\end{adjustwidth}
\hypertarget{16034034156113210739}{} 
\hyperlink{16034034156113210739}{\texttt{Scapin.modal\_strain\_displacement!}}  -- {Method.}

\begin{adjustwidth}{2em}{0pt}


\begin{minted}{julia}
modal_strain_displacement!(B, k, N, h)
\end{minted}

Compute the modal strain-displacement vector \texttt{B} in place, for the spatial frequency \texttt{k}. The grid is defined by its size, \texttt{N} and spacing, \texttt{h}.



\href{https://github.com/sbrisard/Scapin.jl/blob/ea3f90c60e90cdf214e41b1314a8ee608e0b8d10/src/bri17.jl#L1-L6}{\texttt{source}}


\end{adjustwidth}
\hypertarget{10126462544401670939}{} 
\hyperlink{10126462544401670939}{\texttt{Scapin.modal\_strain\_displacement}}  -- {Method.}

\begin{adjustwidth}{2em}{0pt}


\begin{minted}{julia}
modal_strain_displacement(k, N, h)
\end{minted}

Compute the modal strain-displacement vector \texttt{B} for the spatial frequency \texttt{k}. The grid is defined by its size, \texttt{N} and spacing, \texttt{h}.



\href{https://github.com/sbrisard/Scapin.jl/blob/ea3f90c60e90cdf214e41b1314a8ee608e0b8d10/src/bri17.jl#L32-L37}{\texttt{source}}


\end{adjustwidth}
\hypertarget{10268754520622093903}{} 
\hyperlink{10268754520622093903}{\texttt{Scapin.shape}}  -- {Method.}

\begin{adjustwidth}{2em}{0pt}


\begin{minted}{julia}
shape(x, h)
\end{minted}

Return the value of the shape functions of the element, at the specified point.

This function returns a \texttt{d}-dimensional array \texttt{N}, such that \texttt{N[n]} is the shape function associated with node \texttt{n}, evaluated at \texttt{x}. In particular, \texttt{N[n]} evaluated at node \texttt{m} is \texttt{δ[m, n]} (Kronecker).

See “\hyperlink{214055906537151257}{Shape functions}” in the docs.



\href{https://github.com/sbrisard/Scapin.jl/blob/ea3f90c60e90cdf214e41b1314a8ee608e0b8d10/src/bri17.jl#L146-L157}{\texttt{source}}


\end{adjustwidth}
\hypertarget{3239073789057218454}{} 
\hyperlink{3239073789057218454}{\texttt{Scapin.stiffness\_matrix}}  -- {Method.}

\begin{adjustwidth}{2em}{0pt}


\begin{minted}{julia}
global_stiffness_matrix(N, h, μ, ν)
\end{minted}

Return the global stiffness matrix for periodic, homogeneous elasticity.

\begin{itemize}
\item \texttt{N}: grid size


\item \texttt{h}: cell size


\item \texttt{μ}: shear modulus


\item \texttt{ν}: Poisson ratio

\end{itemize}


\href{https://github.com/sbrisard/Scapin.jl/blob/ea3f90c60e90cdf214e41b1314a8ee608e0b8d10/src/bri17.jl#L295-L305}{\texttt{source}}


\end{adjustwidth}
\hypertarget{3699099644231930551}{} 
\hyperlink{3699099644231930551}{\texttt{Scapin.strain\_displacement\_operator}}  -- {Method.}

\begin{adjustwidth}{2em}{0pt}


\begin{minted}{julia}
strain_displacement_operator(x, h)
\end{minted}

Return the strain-displacement operator for the \texttt{d}-dimensional brick element of size \texttt{h}, evaluated at point \texttt{x}.

This function returns a \texttt{(d+3)} dimensional array \texttt{B} of size \texttt{(d, d, 2, …, 2, d)}. If \texttt{n} is the multi-index of the node, and \texttt{i}, \texttt{j}, \texttt{k} are component indices then, the interpolated \texttt{(i, j)} component of the strain at \texttt{x} reads


\begin{lstlisting}
ε[i, j] = Σₙ Σₖ B[i, j, n, k] * u[n, k].
\end{lstlisting}

See “\hyperlink{5967194672630999247}{Gradient and strain-displacement operators}”.



\href{https://github.com/sbrisard/Scapin.jl/blob/ea3f90c60e90cdf214e41b1314a8ee608e0b8d10/src/bri17.jl#L201-L216}{\texttt{source}}


\end{adjustwidth}

\end{document}
